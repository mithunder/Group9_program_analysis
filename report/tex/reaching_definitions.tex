\section{Reaching Definitions}

In this section, Reaching Definitions (RD) analysis will be defined.
RD is a forward-may analysis, that indicates where variables may have
been assigned at any given point. Given that it is a may analysis,
it may include assignments which actually does not reach the given point.
For an imprecise analysis, this is easy to see in the example program;
x := a; abort; skip. An analysis is allowed to let the assignment to x
reach the skip, which is imprecise. To increase the precision of the analysis,
this definition of RD kills the assignment to x before it reaches the skip.

In regards to RD, the definitions is the same as the book given for the course,
except for the kill and gen functions. These will be defined below.
After the definitions, example programs and analysis will be given.

\subsection{Kill}

The definition of kill is the mapping from elementary blocks to the set of variable-label pairs
that are removed by the elementary block. The domain is:
\[final \colon Elementary Command \to \mathcal{P}(Variable)\times\mathcal{P}(Label)\]

kill$_{RD}$(C):\newline
kill$_{RD}$([x := a]$_l$)           = \{(x, ?)\} $\cup$ ((x, l') $\vert$ B$_l$' is an assignment to x in C*)\newline
kill$_{RD}$([read x]$_l$)           = \{(x, ?)\} $\cup$ ((x, l') $\vert$ B$_l$' is an assignment to x in C*)\newline
kill$_{RD}$([skip]$_l$)             =$\emptyset$\newline
kill$_{RD}$([abort]$_l$) 			= ((x, l) $\vert$ l $\in$ labels(C*) and x $\in$ FV(C*) )\newline
kill$_{RD}$([write a]$_l$)          =$\emptyset$\newline
kill$_{RD}$([if]$_l$ gC fi)         =$\emptyset$\newline
kill$_{RD}$([do]$_l$ gC od)         =$\emptyset$\newline
kill$_{RD}$([e]$_l$)                =$\emptyset$\newline

Note that the abort kills everything.

\subsection{Gen}

The definition of gen is the mapping from elementary blocks to the set of variable-label pairs
that are added by the elementary block. The domain is:
\[final \colon Elementary Command \to \mathcal{P}(Variable)\times\mathcal{P}(Label)\]

gen$_{RD}$(C):\newline
gen$_{RD}$([x := a]$_l$)           = \{(x, l)\}\newline
gen$_{RD}$([read x]$_l$)           = \{(x, l)\}\newline
gen$_{RD}$([skip]$_l$)             =$\emptyset$\newline
gen$_{RD}$([abort]$_l$) 			= $\emptyset$\newline
gen$_{RD}$([write a]$_l$)          =$\emptyset$\newline
gen$_{RD}$([if]$_l$ gC fi)         =$\emptyset$\newline
gen$_{RD}$([do]$_l$ gC od)         =$\emptyset$\newline
gen$_{RD}$([e]$_l$)                =$\emptyset$\newline

\subsection{Examples}

Example 1:

\begin{lstlisting}
module rd-example :
read x;[1]
if (x > 0)[2]
  -> y := 0[3]; skip[4]
[] x <= 0[5]
  -> y := 1[6]; abort[7]
fi;
i := 0;[8]
do (i < 10)[9]
  i := i + 1[10]
od
end
\end{lstlisting}
gives as entry for label 8:
\[RD_{entry}(8) = \{(x, 1), (y, 3), (i, ?)\}\]
and exit:
\[RD_{exit}(8) = \{(x, 1), (y, 3), (i, 8)\}\]

For label 10, entry:
\[RD_{entry}(10) = \{(x, 1), (y, 3), (i, 8), (i, 10)\}\]
and exit:
\[RD_{exit}(10) = \{(x, 1), (y, 3), (i, 10)\}\]



