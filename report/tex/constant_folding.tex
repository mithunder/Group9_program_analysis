\section{Constant folding}
Some times programmers decide to declare a constant variable they can use
in their code instead of just writing the value. This tends to make the
code more readable, because ``if x = 2'' gives much less information than
``if x = COLOR\_BLUE''.
  This means that it is not uncommon to find at least a few variables
that can be folded into a constant, particularly in larger programs.
Occasionally it may also be possible to eliminate a branch and therefore
reduce the code size as well.  

\docpar
Constant Folding is a transformation that relies on the information
gathered by the CPA to replace variables with their constant value
if they have one.
  As mentioned in the Constant Propagation section,
CPA can be augmented with an on-the-fly branch elimination transformation
(or alternatively an extended latice to describe the additional case(s)).
The folding transformation will also benefit from augmentation, since it
will make CPA produce more precise results. This will be demonstrated with
this result below.

\begin{lstlisting}
module group9_3_constant_propagation :
       read unknown;[1]
       zero  := unknown * 0;[2]
       one   := zero + 1;[3]
       two   := zero + one * one - zero + one;[4]
       three := one + (two + one) * two - two * two;[5]
       four  := three * two - three + one;[6]
       if (three > two)[7] -> six   := one + two + three[8]
       [] (three < two)[9] -> one  := three[10]
       [] (three = two)[11] -> three := two[12]
       fi;
       seven := six + one;[13]
       write seven;[14]
end
\end{lstlisting}

\docpar
After evaluating [1] through [6], the conclusion is that all variables are constant
with the same value as their name with the exception of unknown, which has some
unknown value. 
  With the branch elimination extension, it would have concluded that [10] and
[12] are unreachable due to the values of the variables three and two in [9] and
[11]. It would also conclude that [8] is always a valid path. Since there are
no other paths through that if statement, it would happily conclude that the
variable six has the constant value 6.
  Without the  branch elimination, things are not quite so good. In fact the
analysis will include data from [10] and [12], which means that it can no
longer conclude the fact that one and three are constants after the loop.
Nor can it say that six is always 6 after the if, because it may be undefined.
  This means that seven in [13] and [14] can only folded to a constant if we
use the branch elimination, otherwise it will be an unknown value.

\docpar
Due to the way we handled abort statements, the CPA had to provide a special
value to denote that a path would always flow through an abort statement to
avoid contaminating its own analysis else where. This had a distinct
advantage that allowed us to extend the Constant Folder to remove some dead
code as well.
  Since CP is a forward must analysis, then if it concludes that entry value of
a statement comes from an abort statement, then this means that all forward paths
here must flow through an abort. However, it is not possible for anything to flow
through an abort during execution of the program, so this actually concludes that
there are no legal path to this statement. Obviously this means the statement
will never be executed and can thus safely be removed.

\begin{lstlisting}
module cf-abort:
       if true[1] -> abort[2]
       fi;
       read x;[3]
       write x[4]
end
\end{lstlisting}

\docpar
Here it is trivial to see that there are no paths through the if without
reaching an abort. The CPA would then give a special value to [3], which
it would propagate to [4] as well. So the folder could trivially conclude
that [3] and [4] were dead.


//TODO: Niels

