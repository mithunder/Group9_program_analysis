
\section{Data Structure}

The program is translated into a tree-like structure where each node
is either a compound node (having children) or is a leaf node representing
an ``3-address expression''.

  A variation ``3-address expression'' is also used in the ``GIMPLE'' trees,
which is the internal/intermediate representation (IR) used in the GNU Compiler
Collection (GCC) ``middle-end''.

\paragraph*{Leaf nodes}
are all ``simple'' expressions, which are rewritten to 1 or 2 operands expressions. Here
simple refers to any non-compound expression (that is all language constructs
except things like do-od, if-fi and \{\})

In this particular IR all expressions are reduced to 1 or 2 operands expressions.
As an example:

\begin{lstlisting}
	c := a * (b + c) / 4
\end{lstlisting}
	
will be translated to the following leaf nodes:

\begin{lstlisting}
	t1 := b + c
	t2 := a * t1
	c := t2/4
\end{lstlisting}
	
In some cases the IR/the language only allows a single node and in these cases
the nodes are wrapped in a single compound (``SCOPE'') node. In some cases the
IR puts SCOPE nodes in places the language would not allow it, as an example:

\begin{lstlisting}
	if ((x > 3) | (y > 3)) -> skip if
\end{lstlisting}

In this case the guard would be represented as:

\begin{lstlisting}
	{
		t1 := x > 3
		t2 := y > 3
		t3 := t1 | t2
	}
\end{lstlisting}

This is not a problem in this language, because when during the reverse transformation
this scope can trivially be replaced with a bracket group and it will now be a valid
program again.

Finally there also some language constructs that may disappear in the IR. Particularly
brackets will vanish. As seen in the example above, the brackets are not represented at
all. They are used when constructing the IR to order leaf node, but 

//TODO: Ending mid-sentence.

\paragraph*{Compound nodes}
refers to language constructs that contains other nodes. In the IR there are 3 kind of
compund nodes, which are IF, DO and SCOPE. A SCOPE node simply contain a list of nodes
in the order of the flow for that scope. The IF and DO nodes are a little more special.

  The IF and DO node have twice as many child nodes as the original if statement in the
language had branches. One of these nodes represents the guard and the other the command.
The nodes are ordered so that guards comes first and the commands follow. The commands
are inserted in the same order as their guard, so the i'th command is protected by the
i'th guard.

\docpar
A special case is the $module x : y end$ construct. This is represented as a ``compilation
unit'' with a root node. The root node is a SCOPE node containing all the nodes that y
represents. This means that (e.g.) x is not a part of the tree structure.

\subsection{Advantages}
One of the advantages of this format is that everything calculation is reduced to
small simple chunks. It is also possible to trivially transform this into a format
that can be executed by ``3-address instruction'' machine.

A second and very important advantage (and disadvantage) is that the flow becomes
a part of the structure. In a SCOPE the control flows from a node to its sibling
and there is no ``magic'' flow in or out of a SCOPE except for the entry at the
first child and the exit at the last child (modulo ABORT statement; see
Disadvantages about that).

  The implicit flow makes it very easy to see whether or not a statement is inside a
``strongly connected component''. This is important because it is if an analysis reaches
a fixed-point inside one of these components, this component is ``done'' (unless there is
a change in a previous component that flows into this component).

  There are two cases in this representation; if the node is a DO, then the node and
everything below is consists of one strongly connected component. Otherwise each leaf
node is its own strongly connected component. Note that IF and SCOPE changes do not
effect the strength of how connected their children are.

\subsection{Disadvantages}
The representation is not trivial to ``untransform'' into the source language even if no
modifications have been made, particularly due to the great amount of temporary variables
introduced. This also has a consequence for some analyses, since their time
or/and space complexity are (partly) bounded by the number of variables.

  A slightly better variation of this IR might have been to use a tree structure to
describe the operands. This representation would likely have its own issues, but would
allow an easier transformation back to source form (since the original expression is now
fully available in the statement) and it would most likely eliminate the need for the
temporary variables. It was planned to change to this IR, but due to time constrains
it was never started.

\docpar
Since the flow is embedded in the structure, non-standard flow is difficult to handle.
As an example if the language had  ``GOTO'' statements, it would have proved complicated.
One could also argue that ABORT does not have a flow; though in this representation the
analysis will end up ``flowing'' through ABORTs.

  A consequence of this is that analyses have to be able to handle ABORT statements and
define an ``exit value'' of an ABORT statement. On the plus side, it turned out that this
exit value for Constant Propagation Analysis could be (ab)used positively.

