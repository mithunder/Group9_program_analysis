\section{Program slicing}

Testing of program slicing has been based on two main methods;
Testing by hand, by running the test and considering if the output
is reasonable, and automated testing. Both in the manual and automated
test source code files in the GCL from the different groups have been
used. This has caused some issues, since program slicing is defined
differently for different groups. Therefore, the expectations of the
programs has changed depending on the definition.
Of note is that program slicing in this project is based on a statement
and not specific variables in a statement.

The program slicing examples and test instructions are given in the 
appendix: \ref{ps_test_examples}

\subsection{Test results}

Different tests has been used, including from other groups in the course
of the project. For those files which the files have been modified or the
expected result is different, it is noted.

When the Notes say ``more precise'', it means that this project's program
slicing is more precise, and likewise for ``less precise''.

\begin{tabular}{ |l||l|p{7cm}| }
\hline
  Name & Result & Notes \\
\hline
  example\_1		& Correct &  \\
\hline
  example\_2		& Correct &  \\
\hline
  def\_prog\_1		& Correct & Taken from the design and implementation \\
\hline
  def\_prog\_2		& Correct & .., control flow \\
\hline
  def\_prog\_3		& Correct & .., do, version where S2 affects P.\\
\hline
  def\_prog\_4		& Correct & .., do, version where S2 does not affect P. \\
\hline
  def\_prog\_5		& Correct & .., if. \\
\hline
  def\_prog\_6		& Correct & .., non-termination. \\
\hline
  group2\_1		& Correct &  Very different definition of program slicing.
Group 2 eliminates control flow fully.\\
\hline
  group3\_1		& Correct &  Different definition of program slicing.
Group 3 removes program structure and flow (if, do). But they include the same
conditions and statements which is present in the result of this projects program slicing,
so they must consider the flow as well, indicating the definition is similar.\\
\hline
  group7\_1		& Correct &  Same comments as for group 3.\\
\hline
  group8\_3		& Correct &  Different expectation; more precise\\
\hline
  group9\_1		& Correct &  This project's group.\\
\hline
  group11\_2		& Correct &  Different expectation; less precise,
but it is a special case regarding if, and the statement itself is not included\\
\hline
  group12\_2		& Correct &  Much more precise; group 12's example includes statements
not affeting the loop or the values computed at the given statement.\\
\hline
  group14\_1		& Correct &  Very different definition of program slicing.
Group 14 eliminates control flow fully.\\
\hline
\end{tabular}

\subsection{Conclusion}

Many different tests, including special cases and general programs,
have been sliced, and the implementation have given the correct output
for each. Based on these results, it is concluded that on several
programs, the program slicing implementation produces the correct output.

It is also noted that several groups have different definitions of program slicing.
It is noted that this project seem to have a more useful definition than some
other definitions.
Group 8 has a very similar definition (the groups have also communicated in regards
to a review), and group 3 and 7 seemed to have definitions a little similar.
Other groups had very different definitions, and investigating the usefulness of
those definitions compared to this projects definition may be fruitful in regards
to choosing a definition.

