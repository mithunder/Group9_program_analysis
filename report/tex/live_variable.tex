
\section{Live variable}

In this section, Live Variable (LV) analysis will be defined.
LV is a backward-may analysis, that indicates which variables are
alive at each command: A variable is alive if there is some path
from the label to a use of the variable that does not redefine it.
A simple example is: x := a; x := b; x := c; write x.
Here, x is dead at the two first commands, since it is redefined
before it is read. Note that variables that are never read will
automatically be dead. By detecting superfluous defines, LV
can be used to remove dead code.

In regards to LV, the definitions is the same as the book given for the course,
except for the kill and gen functions. These will be defined below.
After the definitions, example programs and analysis will be given.

\subsection{Kill}

The definition of kill is the mapping from elementary blocks to the set of variables
that are removed by the elementary block. The domain is:
\[final \colon Elementary Command \to \mathcal{P}(Variable)\]

kill$_{LV}$(C):\newline
kill$_{LV}$([x := a]$_l$)           = \{x\}
kill$_{LV}$([read x]$_l$)           = \{x\}
kill$_{LV}$([skip]$_l$)             = $\emptyset$\newline
kill$_{LV}$([abort]$_l$) 			= FV(C*)\newline
kill$_{LV}$([write a]$_l$)          =$\emptyset$\newline
kill$_{LV}$([if]$_l$ gC fi)         =$\emptyset$\newline
kill$_{LV}$([do]$_l$ gC od)         =$\emptyset$\newline
kill$_{LV}$([e]$_l$)                =$\emptyset$\newline

Note that the abort kills everything, since x will clearly not survive to the write in
x := a; abort; write x, and is therefore dead at the write.

\subsection{Gen}

The definition of gen is the mapping from elementary blocks to the set of variables
that are added by the elementary block. The domain is:
\[final \colon Elementary Command \to \mathcal{P}(Variable)\]

gen$_{LV}$(C):\newline
gen$_{LV}$([x := a]$_l$)           = FV(a)\newline
gen$_{LV}$([read x]$_l$)           = $\emptyset$\newline
gen$_{LV}$([skip]$_l$)             = $\emptyset$\newline
gen$_{LV}$([abort]$_l$) 		   = $\emptyset$\newline
gen$_{LV}$([write a]$_l$)          =FV(a)$\emptyset$\newline
gen$_{LV}$([if]$_l$ gC fi)         =$\emptyset$\newline
gen$_{LV}$([do]$_l$ gC od)         =$\emptyset$\newline
gen$_{LV}$([e]$_l$)                =FV(e)$\emptyset$\newline

\subsection{Examples}

Example 1:
\begin{lstlisting}
module rd-example-1 :
x := 10;[1]
x := 5000;[2]
x := a;[3]
write y[4]
\end{lstlisting}
gives for all entries:
\[LV_{entry}(all) = \{y\}\]
and exit, except 4:
\[LV_{exit}(all-4) = \{y\}\]
Four gives:
\[LV_{exit}(4) = \{\}\]
since nothing reads y after the last statement.

Example 2:
\begin{lstlisting}
module rd-example-2 :
x := z;[1]
if (z > 0)[2] -> x := -1[3]
[] (z <= 0)[4] -> x := 1[5]
fi;
write x*127[6]
\end{lstlisting}
A little more advanced example.
For label 1:
\[LV_{entry}(1) = \{z\}\]
Since x is never used before assignment, and
\[LV_{exit}(1) = \{z\}\]
since x is never used before being assigned to again.

For label 3:
\[LV_{entry}(3) = \{\}\],
\[LV_{exit}(3) = \{x\}\]
since x is used after the assignment without being define again,
namely at label 6. z, however, is never used again.

