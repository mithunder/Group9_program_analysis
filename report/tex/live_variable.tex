
\section{Live variable}

//TODO: Niels

In this section, Live Variable (LV) analysis (LVA) will be defined.
LVA is a backwards-may analysis, that indicates which variables are
alive at each command: A variable is alive if there is some path
from the label to a use of the variable that does not redefine it.

  A simple example is: x := a; x := b; x := c; write x.
Here, x is dead at the two first commands, since it is redefined
before it is read. Note that variables that are never read will
automatically be dead. By detecting superfluous defines, LV
can be used to remove dead code.

In regards to LV, the definitions is the same as the book given for the course,
except for the kill and gen functions. These will be defined below.
After the definitions, example programs and analysis will be given.

\subsection{Kill}

The definition of kill is the mapping from elementary blocks to the set of variables
that are removed by the elementary block. The domain is:
\[final \colon Elementary Command \to \mathcal{P}(Variable)\]

\docpar
kill$_{LV}$(C):\newline
kill$_{LV}$([x := a]$_l$)           = \{x\}
kill$_{LV}$([read x]$_l$)           = \{x\}
kill$_{LV}$([skip]$_l$)             = $\emptyset$\newline
kill$_{LV}$([abort]$_l$) 			= FV(C*)\newline
kill$_{LV}$([write a]$_l$)          =$\emptyset$\newline
kill$_{LV}$([if]$_l$ gC fi)         =$\emptyset$\newline
kill$_{LV}$([do]$_l$ gC od)         =$\emptyset$\newline
kill$_{LV}$([e]$_l$)                =$\emptyset$\newline

Note that the abort kills everything, since x will clearly not survive to the write in
x := a; abort; write x, and is therefore dead at the write.

\subsection{Gen}

The definition of gen is the mapping from elementary blocks to the set of variables
that are added by the elementary block. The domain is:
\[final \colon Elementary Command \to \mathcal{P}(Variable)\]

\docpar
gen$_{LV}$(C):\newline
gen$_{LV}$([x := a]$_l$)           = FV(a)\newline
gen$_{LV}$([read x]$_l$)           = $\emptyset$\newline
gen$_{LV}$([skip]$_l$)             = $\emptyset$\newline
gen$_{LV}$([abort]$_l$) 		   = $\emptyset$\newline
gen$_{LV}$([write a]$_l$)          = FV(a)\newline
gen$_{LV}$([if]$_l$ gC fi)         = $\emptyset$\newline
gen$_{LV}$([do]$_l$ gC od)         = $\emptyset$\newline
gen$_{LV}$([e]$_l$)                = FV(e)\newline

\subsection{LV, Faint variables and Strongly Live Variables}
LVA is like RDA a member of the bit-vector analyses. This has some advantages
in terms of simplicity, such as all operations can be specified as trivial
kill and gen functions.
  Unfortunately it turns out this simplicity of LVA has a price, namely that
it sometimes considers variables live that are in fact dead. These variables
are referred to as ``Faint Variables''.
  Faint variables appear because gen does not consider the relevance of the
variable being assigned to; thus it will always consider all variables in
the expression to be alive. Non-faint live variables are sometimes also referred
to as ``Strongly Live Variables''.

\docpar
Trivial analysis of LVA shows that it is possible to upgrade the algorithm into
ignoring faint variables; this upgraded algorithm is sometimes called ``Strongly
Live Variables Analysis'' (SLVA). The upgrade is conditionally using gen on
assignment statements; that is:

\docpar
gen$_{LV}$([x := a]$_l$)           = FV(a)\newline

\docpar
becomes:

\docpar
gen$_{LV}$([x := a]$_l$)           = $$\begin{cases}
$FV(a) if x is live$\\
\emptyset$ else$
\end{cases}
$$\newline

\docpar
The unfortunate side effect of this is that SLVA is not a true bit-vector
analysis. Nevertheless, even with the change it is still a part of the
monotone framework, so using SLVA is not an issue.

\subsection{Examples}
These examples will demonstrate how LVA and SVLA work; for the sake of
clarification strongly live variables will be written in bold
(e.g. \textbf{y}) to separate them from the faint variables.

\docpar
Example 1:
\begin{lstlisting}
module lv-example-1 :
x := 10;[1]
x := 5000;[2]
x := a;[3]
write y[4]
\end{lstlisting}
The entry value of [4] is the empty set; its
exit value is 
\[ SLV_{exit}(4) = LV_{exit}(4) = \{\textbf{y}\}\]
The entry value of [3] is trivially equal to the
exit value of [4]. But in [3] LVA is unable to see
that since x is dead, then a should not be
generated. So a is added as a faint variable:

\[LV_{exit}(3) = \{a, \textbf{y}\}\]

On the other hand SLVA will be able to tell and thus
when using SLVA, the exit value will be:

\[SLV_{exit}(3) = \{\textbf{y}\}\]

Continuing with the analysis it turns out that the
entry and the exit value of [1] and [2] is just the
exit value of [3]. Looking at the results then the
data from LVA makes it possible to remove [3].
In comparison the data from SLVA will also mark [1]
and [2] as redundant.
  After removing [3], it is possible to rerun LVA
and conclude now that [1] and [2] are also redundant.

\docpar
Example 2:
\begin{lstlisting}
module lv-example-2 :
x := z;[1]
if (z > 0)[2] -> x := -1[3]
[] (z <= 0)[4] -> x := 1[5]
fi;
write x*127[6]
\end{lstlisting}


