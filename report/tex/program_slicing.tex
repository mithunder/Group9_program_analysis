\section{Program slicing}

The assignment defines program slicing  as:
``Given a program the idea is to determine the part of the program that may influence the values
computed at a given point of interest; this part of the program is then called a program slice.''
Based on this description, program slicing as understood in this project is defined and investigated,
to try and obtain a useful, safe and precise program transformation.
One use of program slicing is to use it for debugging; for instance, given a program that handles
multiple concerns, isolate the part of the program that just gave a divide-by-zero error at a
specific statement.

In regards to approximation, the program slicing is allowed to overapproximate: It is allowed to include statements
that falsely indicate that more values can be computed than what really can be computed.
The main requirement is that it does not exclude statements that would result in an underapproximation
of values of what may be computed. Of course, to enable a more precise transformation, it is desirable
to minimise the overapproximation.

To illustrate what program slicing is, a simple example is given:
\begin{lstlisting}
a := 3;
b := 5;
x := a;
\end{lstlisting}
With point of interest at line 3, this becomes:
\begin{lstlisting}
a := 3;
x := a;
\end{lstlisting}
since line 2 has no effect on the values computed at line 3.
This is the only parts of the program that are considered interesting.

\subsection{Control flow}

An important issue regarding program slicing is that not only assignment, but also program control
flow is important. For instance, consider the following example:
\begin{lstlisting}
a := 3;
b := 4;
if (b = 4) -> a := 5;
[] (b = 3) -> a := 10000;
fi;
x := a;
\end{lstlisting}
With point of interest at line 6.
If control flow is ignored, the program slice might end up looking like this:
\begin{lstlisting}
a := 5;
a := 10000;
x := a;
\end{lstlisting}
which is clearly wrong. Instead, the correct program slice must be:
\begin{lstlisting}
b := 4;
if (b = 4) -> a := 5;
[] (b = 3) -> a := 10000;
fi;
x := a;
\end{lstlisting}
since this includes all the parts that are relevant to determine the parts
that may influence the values at line 6.
Note that b is included, since it helps determine the flow: it is included in at least one of the
conditions in the guarded if. Furthermore, the first assignment to a is gone, since at least one of the
statements in line 3 and 4 is executed.
Based on these considerations, control flow must be included, at least to some degree.

To consider control flow properly, both the guarded if and the guarded do will be considered. Early
termination and non-termination is also considered.

\subsection{Guarded do}

For the guarded do, just the conditions with commands that includes definitions which reaches
referenced variables are included in the program slice. This claim is justified below.

Consider some program:
\begin{lstlisting}
a := 42;
do (P) -> a := 30; S1;
[] (Q) -> S2;
od
x := a;
\end{lstlisting}
and that the point of interest is the last command.

Assume that S2 does not contain any assignments to a, and that no definitions in S2 will reach a.
It is clear that the predicate P is relevant to a,
and therefore to x. It should therefore be included in the program slice. Instead, the relevance of Q
$\rightarrow$ S2 is not determined yet. For S2, there are two cases: If it causes definitions that reaches P, it is
relevant, and would be included in a use-definition run on the variables of P at P. If it instead has no
definitions that reaches P, it is irrelevant to the values of a. No matter if it is true or false, the do will
still be executed as long as P is true, and since S2 does not touch P, it has no effect on how many times
a := 30; S1; is executed. Therefore, the correct program slice is:
\begin{lstlisting}
a := 42;
do (P) -> a := 30; S1;
od
5: x := a;
\end{lstlisting}
Note that only the parts of S1 affecting P or a should be included.
Based on this, for guarded do's, just the conditions with commands that includes definitions which
reaches referenced variables are included in the program slice.

\subsection{Guarded if}

For the guarded if, all the conditions should be included in the program slice if any definition in any
of their commands reaches the referenced variables. This claim is justified below:
Consider 2 different programs:
\begin{lstlisting}
a := 2;
if (P) -> a := 3;
[] (Q) -> a := 4;
[] (R) -> z := 0;
fi;
x := a;
\end{lstlisting}
Here, it may seem clear that line 4 is irrelevant. But it is in fact still relevant. Assume that line 4 is
removed in the program slice. In this case, the definition from line 1 will never reach line 6, since it
will either be overwritten in line 2 and 3, or the program will terminate before line 6. But this is
different from the semantics of the original program; here, the definition from line 1 does reach line
6, since line 4 does not assign to a.
Based on this, for guarded if's, all the conditions should be included in the program slice if any
definition in any of their commands reaches the referenced variables.

\subsection{Non-termination and early termination}

Non-termination and early termination may have an effect on whether or not the point of interest is
ever reached, but not directly which values that may be computed. Therefore, non-termination and
early termination may be ignored, as long as it does not result in an underapproximation of the
values that may be computed. For instance, look at the following program:
\begin{lstlisting}
a := 4;
do (true) -> skip
od;
x := a;
\end{lstlisting}
The correct program slice would be:
\begin{lstlisting}
a := 4;
x := a;
\end{lstlisting}
Note that this is also a very practical definition, partly because non-termination is generally
undecidable.

\subsection{Corner cases}

A corner case of the program is that a condition may be included without its command is included.
This is the case for the guarded if, since all conditions are included whether or not there is
something relevant in their commands. For instance,
\begin{lstlisting}
a := 2;
if (P) -> a := 3
[] (R) -> z := 0
fi;
x := a;
\end{lstlisting}
would end up looking like:
\begin{lstlisting}
a := 2;
if (P) -> a := 3
[] (R) ->
fi;
x := a;
\end{lstlisting}
This is not legal semantics. In these cases, the algorithm should repair this “damaged” guarded
statement by providing a skip statement, like this:
\begin{lstlisting}
a := 2;
if (P) -> a := 3
[] (R) -> skip
fi;
x := a;
\end{lstlisting}

\subsection{Design of algorithm}

Now, an algorithm is designed. The basis of the algorithm is to have a queue of commands to
investigate. It starts out with adding the points of interest to the queue. Then, for the next
item on the queue, if the command has not been investigated, it is investigated. Based on that,
more commands may be added to the investigation queue. The commands to investigate is based
on two questions: What definitions are used at this statement, and what parent does this statement
have? Namely, all reaching definitions (using RD-analysis) that are used at this command is added
to the investigation queue, and all parents of this command is added.

In regards to special handling of the commands, if-commands get all their conditions added,
while do-loops does no such things. For each guard-command pair, the guard is added to the queue
if the command is investigated.

In pseudo-code:
\begin{lstlisting}
Initialize
Add points of interest
while (!queue.empty())
	currentStatement = queue.poll();
	if (!isIncluded(currentStatement))
		investigateValues(queue, currentStatement, RDAnalysis)
		investigateParents(queue, currentStatement)
		investigateSpecialStatements(queue, currentStatement)
Remove all parts not included
Cleanup broken parts
\end{lstlisting}

