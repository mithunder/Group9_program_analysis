\section{General definitions}
When arguing for the correctness of algorithms, defining the flow and labels of the language is
valuable. Below different definitions and functions is given, mirroring those of chapter 2 of the
book given in the course. These include init(C), final(C), blocks(C), labels(C) and flow(C).
Note that C is used instead of S to indicate commands instead of statements, and that
x and y typically denotes variables, and a and e expressions, though this is not always followed.

In the following, the guarded commands are a part of the domain of commands.

\subsection{Termination}

An important point in regards to the flow is whether or not the program is terminated
at a certain point. For instance, the program may terminate at an if, but is not
certain to do so. On the other hand, an abort is certain to terminate, and no flow
occurs in a command that looks like: abort; C$_2$. To get a precise analysis, termination
may be considered. However, since this represents a complication, abort is not guaranteed
to be considered everywhere. In the cases where abort is not considered, the below analysis
can still be applied: replace any occurrence of ``abort'' in the program with ``skip''.
This simplifies the analysis, but makes it less precise.

\subsection{Labelling}

A label is given to every elementary block ``el''. The elementary blocks consists of
the commands assign, skip, abort, read, write, and test. Furthermore, a label is given
to if and do, to simplify definitions later. For all intentions and purposes, the
label of if and do can be seen as being the label of a skip elementary block.

\subsection{Helper functions}

Some helper functions:

guards([e]$_l$ $\to$ C)      = \{[e]$_l$\} \newline
guards(gC$_1$ [] gC$_2$)= guards(gC$_1$) $\cup$ guards(gC$_2$)\newline

\subsection{Init}

init(C) denotes the entry point for the flow of a command.
For instance, the entry point of C$_1$ ; C$_2$ is equal to init(C$_1$),
since C$_1$ contains the entry point. The domain is:
\[init \colon Command \to Label\]
It should be noted that the domain is different from that of the book.

init(C):\newline
init([x := a]$_l$)      = l\newline
init([skip]$_l$)        = l\newline
init([abort]$_l$)       = l\newline
init([read x]$_l$)      = l\newline
init([write a]$_l$)     = l\newline
init(C$_1$; C$_2$)        = init(C$_1$)\newline
init(\{C\})             = init(C)\newline
init([if]$_l$ gC fi)        = l\newline
init([do]$_l$ gC od)        = l\newline

The reasoning behind not simply defining the entry point of an if or a do as the first
test in that if or do is that not only the first test, but all the tests, are entry points.
Thus, simply choosing the first label would be wrong. One way to include all tests would
be to extend the definition of init to map to the powerset of all labels, but this
can complicate the analysis later. Instead, all the tests are included by labelling
the if and do.



\subsection{Final}

final(C) denotes the exit points for the flow of a command.
For instance, the exit point of [read x]$_l$ is l. The domain is:
\[final \colon Command \to \mathcal{P}(Label)\]

\docpar

$final(C):$\\
$final([x := a]_l)     = \{l\}$\\
$final([skip]_l)       = \{l\}$\\
$final([abort]_l)      = \{l\}$\\
$final([read x]_l)     = \{l\}$\\
$final([write a]_l)    = \{l\}$\\
$final(C_1; C_2)		  =
\begin{cases}
(el \vert el \in final(C_1) \wedge el = abort) \cup final(C_2) \textbf{if} (el \vert el \in final(C_1) \wedge el \neq abort) \neq \emptyset \\
(el \vert \in final(C_1) \wedge el = abort) \textbf{else}
\end{cases}$\\
$final(\{C\})          = final(C)$\\
$final(if gC fi)       = final(gC)$\\
$final([do]_l gC od)  = 
guards(gC) \cup (el \vert el \in final(gC) \wedge el = abort)$\\
$final([e]_l \to C)    = final(C)$\\
$final(gC1 [] gC2)      = final(gC1) \cup final(gC2)$\\

\begin{itemize}
\item The reasoning behind the definition of the semicolon command is that the
final commands of C$_2$ will only be reached if there is any flow at all
between C$_1$ and C$_2$: this is not the case if C$_1$ aborts the program
before reaching C$_2$.
\item The reasoning behind the definition of final for if is that the exit points
of an if consists of the final commands in all the commands of the if.
\item The reasoning behind the definition of final for do is that the exit points
of a do consists of all the tests in the do (represented by [do]$_l$)
as well as the final commands that aborts, since the normal final commands
simply flow back to the tests.
\end{itemize}



\subsection{Blocks}

blocks(C) denotes the elementary blocks of a command.
For instance, the elementary blocks of blocks([read x]$_l$) is l. The domain is:
\[final \colon Command \to \mathcal{P}(Elementary Command)\]

\docpar
\begin{equation}
blocks(C):\newline
blocks([x := a]_l)      = {[x := a]_l}\newline
blocks([skip]_l)        = {[skip]_l}\newline
blocks([abort]_l)       = {[abort]_l}\newline
blocks([read x]_l)      = {[read x]_l}\newline
blocks([write a]_l)     = {[write a]_l}\newline
blocks(C_1; C_2)		 = 
\begin{cases}
blocks(C_1) \cup blocks(C_2) \textbf{if} (flow(C_1; C_2) \setminus (flow(C_1) \cup flow(C_2))) \neq \emptyset\\
blocks(C_1) \textbf{else}
\end{cases}
\newline
blocks(\{C\})             = blocks(C)\newline
blocks(if gC fi)        = blocks(gC)\newline
blocks(do gC od)        = blocks(gC)\newline
blocks([e]_l \to C)      = \{[e]_l\} \cup blocks(C)\newline
blocks(gC_1 [] gC_2)= blocks(gC_1) \cup blocks(gC_2)\newline
\end{equation}

The reasoning behind the definition of the semicolon command is that the
blocks function only includes those commands that may be reached. If there is
no flow from C$_1$ to C$_2$, the blocks in C$_2$ will never be reached, and is
therefore never included.



\subsection{Flow}

Flow denotes the possible transitions between elementary blocks.
For instance, the flow of flow(skip[l$_1$]; skip[l$_2$]) is equal to
(l$_1$, l$_2$), since the program would simply transition from the first
skip to the last skip. The domain is:
\[flow \colon Command \to \mathcal{P}(Label)\times\mathcal{P}(Label)\]

flow(C):\newline
flow([x := a]$_l$)        =$\emptyset$\newline
flow([skip]$_l$)          =$\emptyset$\newline
flow([abort]$_l$)         =$\emptyset$\newline
flow([read x]$_l$)        =$\emptyset$\newline
flow([write a]$_l$)       =$\emptyset$\newline
flow\-between(C$_1$, C$_2$)= ((el$_1$, el$_2$) $\vert$ el$_1$ $\in$ final(C$_1$) $\wedge$ el$_1$ $\neq$ abort $\wedge$ el$_2$ $\in$ init(C$_2$))\newline
flow(C$_1$; C$_2$)		 = 
\begin{equation}
\begin{cases}
flow(C_1) \cup flow(C_2) \cup flow-between(C_1, C_2) \textbf{if} flow-between(C_1, C_2) \not = \emptyset\\
flow(C_1) \textbf{else}
\end{cases}
\end{equation}\newline
flow(\{C\}) = C\newline

flow([if]$_l$ gC fi) = ((el$_1$, el$_2$) $\vert$ el$_1$ = [if]$_l$ $\wedge$ el$_2$ $\in$ guards(gC))\newline
$\cup$ flow(gC)\newline

flow([do]$_l$ gC od) = ((el$_1$, el$_2$) $\vert$ el$_1$ = [do]$_l$ $\wedge$ el$_2$ $\in$ guards(gC))\newline
$\cup$ flow(gC)\newline
$\cup$ ((el$_1$, el$_2$) $\vert$ el$_1$ $\in$ final(gC) $\wedge$ el$_1$ $\neq$ abort $\wedge$ el$_2$ $\in$ guards(gC))\newline

flow([e]$_l$ $\to$ C)      = \{([e]$_l$, init(C))\} $\cup$ flow(C) \newline
flow(gC$_1$ [] gC$_2$)= flow(gC$_1$) $\cup$ flow(gC$_2$)\newline


\begin{itemize}
\item The reasoning behind the definition of the semicolon command is that there
is only flow from C$_1$ to C$_2$ if C$_1$ does not abort in all final statements.
\item The reasoning behind the definition of flow for if is that the flow should
flow through the guards, from each guard to its command, and flow through the command.
\item The reasoning behind the definition of flow for do is that the flow should flow
like for if, except that it should flow back to the do label. This is because the do loops,
unlike the if.
\end{itemize}

