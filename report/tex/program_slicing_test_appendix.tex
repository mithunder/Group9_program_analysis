
\subsection{Program slicing testing instructions and examples}
\label{ps_test_examples}

Apart from running the tests, in order to run program slicing
on a source code file, an "annotation" must be given. An annotation in the program
is a line starting with @, followed by some annotation.
The annotation for program slicing is @programslicing=``true''.
An example input program is given below:

\begin{lstlisting}
module example_1 :
  read x;
  read y;
  x := 0;
@programslicing="true"
  z := x / y;
  write z
end
\end{lstlisting}

This will give the output:

\begin{lstlisting}
module example_1 :
  read y;
  x := 0;
@programslicing="true"
  z := x / y
end
\end{lstlisting}

In fact, multiple annotations of program slicing can be used together
to determine the program slice from multiple statements:

\begin{lstlisting}
module example_2 :
  read x;
  read y;
  do
    x > 0 ->
      x := x - 1
    [] x < 0 ->
      x := x + 1
  od;
  sum := 0;
  i := 0;
  do
    i < 10 ->
      sum := sum + i;
@programslicing="true"
      i := i + 1
  od;
  x := 0;
@programslicing="true"
  z := x / y;
  write z
end
\end{lstlisting}

Giving:

\begin{lstlisting}
module example_2 :
  read y;
  i := 0;
  do
    i < 10 ->
@programslicing="true"
      i := i + 1
  od;
  x := 0;
@programslicing="true"
  z := x / y
end
\end{lstlisting}

As can be seen, the sum has been removed, both the slice for i
and the slice for z has been isolated properly.

